Quotes:

Genome project

Mick Liubinskas, serial entrepreneur and co-founder of Pollenizer

“Scaling at the right time is tough. Too early and you waste money and get
distracted. Too late and you miss the market or run out of runway. "


Alexander Galitsky, serial entrepreneur and investor

"Founders often forget that approximately >200k early adopters exist around the
globe. Their goal is to try every new product - but not to use or pay for it. Therefore
200K free users mean very little. Instead of scaling your user base prematurely
good startups are focusing on small defined user groups."



Genome:

"to crack the innovation code of Silicon Valley", 

"...large companies are excellent at sustaining
innovation but by and large fail at disruptive innovation. Startups thrive on
creating disruptive innovations. Recently, thought leaders in entrepreneurship
have come to the conclusion that in order for large companies to be effective at
disruptive innovation they need to make structural changes that make them
behave nearly identically to startups."


http://blogs.hbr.org/hbr/hbreditors/2012/12/lifes_work_2012_hbr_interviews.html

 Muhammad Yunus about scaling up. "I look just at one plot, not the whole plantation. 
 I do the plot and it works, so I do the next plot the same way.
 You start with 100 people and then move to the next 100 people... 
 you're adding up to a bigger scale at a gradual speed. 
 Then you have to monitor and start linking the structure and so on. 
 But you're not designing at the outset for a million people, starting with a megastructure. 
 You're moving step by step."

---

http://hbswk.hbs.edu/item/5170.html
Clayton M. Christensen: "What Customers Want from Your Products"
"With few exceptions, every job people need or want to do has a social, a functional, and an emotional dimension."
"The job, not the customer, is the fundamental unit of analysis."
"Job-defined markets are generally much larger than product category-defined markets. "
---
"It's far easier to overshoot by adding functionality to a simple product than it is to take cost or functionality out of a complex offering. "
"The power lies in clarity."
How to Disrupt an Overlooked Market
http://www.businessweek.com/stories/2011-06-02/how-to-disrupt-an-overlooked-marketbusinessweek-business-news-stock-market-and-financial-adviced
---
Modified definition of DI:
"a phenomenon in which a new product (including service, process, and business
model) replaces existing dominant design with exceptional commercial success." \cite{managingDisruptiveInnovation}
---


----
  \emph{New Venture Performance}\\
  
"Risk management involves keeping failures small and having them happen early, and
then building upon them for future success."

"The essence of effectuation is the use of
nonpredictive strategies including the affordable loss principle"

--
effectuationBook
\emph{EFFECTUATION: ELEMENTS OF ENTREPRENEURIAL EXPERTISE}\\

  "failing is an integral part of venturing well. ... 
    outlive failures by keeping them small and killing them young"



\emph{Sarasvathy - effectuation vs. causation}\\
  http://www.youtube.com/watch?v=hCMpd7z4AbA

  "when the product, business model, distribution network work, you need causal to scale"
  "effectuation: invent the wheel; causal not reinventing the wheel but scale"
  scientific: if you know and can predict, no need effectuation.
  
  Most value: bring people on board.
  
  
"both the effectuation and the causation models can be suitable for the
development and creation of new products and markets" \cite{effectuationCausationEffectuationAndPragmatism}


"Second, effectuation typically only works for evolutionary development, while for
radical disruptive innovations the causation model is more appropriate. Since the effectuation
model starts from what is already there and gradually develops this into something new, it
hinders the development of revolutionary changes. For such changes, vision, long term goals,
anticipation of customer needs, and thinking beyond what is currently possible are important
(Walsh, 2004). Hence, both the effectuation and the causation models can be suitable for the
development and creation of new products and markets." \cite{}

"New Technology-Based Firms in the New Millenium", Chapter 13:  The Nature of the Entrepreneurial Process: Causation, Effectuation, and Pragmatism
2012
% http://www.emeraldinsight.com/books.htm?issn=1876-0228&volume=9&PHPSESSID=5quega043s5ab8tjno37jo9220

Link to effectuation
"Therefore, there is a good reason to speculate
that the entrepreneurship literature may offer contributions to advance the disruptive innovation
research and offer valuable guidance to managers. However, as shown in the table, to-date the direct
linkage between the two streams of research has been minimal." \cite{managingDisruptiveInnovation}


--
"Exaptation process is more effective in generating disruptive innovation than adaptation
 process"  \cite{managingDisruptiveInnovation}
Exaptation: "a trait can evolve because it served one particular function, but subsequently it may come to serve another."


 "Predicting which artifacts or design will turn into the next disruptive innovation is not an easy task
due to the presence of Knightian uncertainty (unknown and unknowable conditions) of plausible
product/design features combinations and permutations. "

TATA: not known, can target a given market and end elsewhere.

--
\cite{managingDisruptiveInnovation}

"Lead user innovation is an effective approach for leveraging outsiders and tap into their innovative
ideas. Although the literature tends to report only the incremental innovation potential of lead users,
we argue that the literature has overlooked the lead users such as the founders of Facebook, ..."

" can leverage the communities of users of their products and learn to distinguish and
reward top ideas and artifacts through company-based initiatives or even leveraging external online
platforms that focus on this."

TATA: engage the users, partnership. Forums?


BUSINESS MODEL
--
\cite{disruptiveInnovationToBusinessModelInnovation}
  "a business model concerns with how a firm creates value for customers and how it captures some of the value"

Genome Report, premature scaling:

 "No one takes venture money to stay a small business "

First paper Effectuation 2001:
-------------------------------

 "The generalized end goal or aspiration remains the same both in causation and effectuation."
"...an effect is the operationalization of an abstract human aspiration."
"The [difference] is in the set of choices: choosing between means to create a particular effect, versus choosing between many 
possible effects using a particular set of means." Many-to-one Vs. one-to-many mappings.
"Both [] are integral parts of human reasoning that can occur simultaneously"
(traditional)
 analytical => current/traditional marketing. 
 "proceed inward to specifics from a larget, general universe-
  That is, to an optimal target segment from a predetermined market"

  "Causation processes are effect dependent"
  "Effectuation processes are actor dependent"
  
  *risk: known distribution, uncertainty: unknown distribution
  ".. real life examples of uncertainty include dealing with ...commertialization of [radical] innovations"
  
  "iN [CAUSATION] THE MARKET IS ASSUMED TO exist independent of the firm... and the task becomes to 
  grab as much of that market as possible. In [effectuation] the founder, along with others, creates the market by bringing together 
  enough stakeholders who 'buy into' the idea to sustain the enterprise"

  

 
-- PAPER: EFFECTUATION ---
"In the case of effectuation, the manufacturer has a general idea that
might lead to a product that could be marketed profitably."


------------------------------
http://www.paulgraham.com/start.html
Note: very interesting read. Re-read and take more notes. Experienced entr. is talking!

"People who don't want to get dragged into some kind of work often develop a protective incompetence at it...
most husbands use the ... trick to some degree."


http://www.paulgraham.com/growth.html#f14n

"But acquirers have an additional reason to want startups. A rapidly growing company is not merely valuable, but dangerous. If it keeps expanding, it might expand into the acquirer's own territory. Most product acquisitions have some component of fear. Even if an acquirer isn't threatened by the startup itself, they might be alarmed at the thought of what a competitor could do with it. And because startups are in this sense doubly valuable to acquirers, acquirers will often pay more than an ordinary investor would. [14]"


